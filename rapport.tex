\documentclass[11pt,a4paper]{article}

\usepackage[utf8]{inputenc}
\usepackage[frenchb]{babel}
\usepackage{lmodern}
\usepackage[T1]{fontenc}

\usepackage{listingsutf8}
\usepackage{graphicx}
\usepackage{xcolor}
\usepackage{fullpage}
\usepackage{hyperref}

\title{Rapport d'objets dupliqués}
\author{Korantin Auguste \& Maxime Arthaud}
\date{\today}

\begin{document}

\maketitle

\section{Introduction}

À première vue, le sujet nous paraissait relativement intéressant: concevoir un système gérant l'état d'objets est plutôt intéressant,
et il y avait l'air d'y avoir pas mal de challenge au niveau de la concurrence et des problèmes de deadlock, etc...

\section{Phase 1}

Le code est venu plutôt naturellement à partir du squelette fourni.

Nous avons rapidement abouti à une version fonctionnelle, mais après développement d'un «~fuzzer~» pour torturer un peu notre système,
plein de bugs/deadlock sont apparus.
Même si la plupart ont pu être résolus assez rapidement, certains apparaissaient dans des
cas très vicieux et ont été particulièrement ardus à résoudre.

\section{Phase 2}

Nous avons commencé par envisager l'utilisation de l'API «~reflect~» de Java, pour créer une classe proxy qui serait capable d'exposer
les fonctions originales de manières transparentes, mais avons fini par abandonné devant certains problèmes très difficiles à surmonter
(problèmes de type, principalement).

Nous nous sommes donc rabattus sur de la génération de code bête et méchante, qui a l'avantage d'être très simple. Mais c'est sale.

\section{Phase 3}

La phase 3 a consisté principalement à rajouter une méthode pour sérialiser des \verb+SharedObject+, ce qui n'a pas posé de difficulté
particulière, même si des tests ont pu montrer qu'il y avait quelques corrections annexes à effectuer pour que tout fonctionne bien.

\section{Conclusion}

Le projet n'a pas été toujours évident, principalement à cause de la gestion de la concurrence dans tous les cas tordus
qui peuvent arriver en montant en charge. Au final, nous pensons toutefois avoir quelque chose de très robuste !

Le reste n'a pas vraiment posé problème. Ah si: on aime pas le Java, mais encore sur ce projet, ça allait (pas comme le JEE, horreur !).

\end{document}
